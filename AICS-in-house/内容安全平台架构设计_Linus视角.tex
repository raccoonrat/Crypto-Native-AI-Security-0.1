%%%%%%%% 内容安全平台架构设计 %%%%%%%%%%%%%%%%%

\documentclass{article}

% 中文支持
\usepackage[UTF8]{ctex}
\usepackage{xeCJK}

% 图表和排版
\usepackage{microtype}
\usepackage{graphicx}
\usepackage{subfigure}
\usepackage{booktabs}
\usepackage{algorithm}
\usepackage{algorithmic}
\usepackage{tikz}
\usetikzlibrary{shapes,arrows,positioning,decorations.pathreplacing,calc}

% 超链接
\usepackage{hyperref}
\hypersetup{
    colorlinks=true,
    linkcolor=blue,
    filecolor=magenta,      
    urlcolor=cyan,
}

% 数学公式
\usepackage{amsmath}
\usepackage{amssymb}
\usepackage{mathtools}
\usepackage{amsthm}

% 代码高亮
\usepackage{listings}
\usepackage{xcolor}

% 代码样式
\lstset{
    language=C,
    basicstyle=\ttfamily\small,
    keywordstyle=\color{blue}\bfseries,
    commentstyle=\color{green!60!black},
    stringstyle=\color{red},
    numbers=left,
    numberstyle=\tiny\color{gray},
    stepnumber=1,
    numbersep=5pt,
    frame=single,
    breaklines=true,
    breakatwhitespace=true,
    tabsize=4,
    showstringspaces=false
}

% 定理环境
\theoremstyle{plain}
\newtheorem{theorem}{定理}[section]
\newtheorem{proposition}[theorem]{命题}
\newtheorem{lemma}[theorem]{引理}
\newtheorem{corollary}[theorem]{推论}
\theoremstyle{definition}
\newtheorem{definition}[theorem]{定义}
\newtheorem{principle}[theorem]{原则}
\theoremstyle{remark}
\newtheorem{remark}[theorem]{注记}

% 页面设置
\usepackage{geometry}
\geometry{a4paper,left=2.5cm,right=2.5cm,top=3cm,bottom=3cm}

% 标题信息
\title{内容安全平台架构设计:\\代码质量与风险分析}
\author{}
\date{2025年1月}

\begin{document}

\maketitle

\begin{abstract}
本文对构建支持A/B测试的In-house内容安全平台进行架构设计与代码质量分析。核心观点是:\textbf{简单优于复杂,代码优于配置,确定性优于随机}。我们批判性地分析了现有四层漏斗架构的过度工程化问题,提出了简化的三层架构(Gateway + Fast Model + Deep Model),并强调统一接口设计原则:让特殊情况消失,而不是增加条件判断。本文详细阐述了统一接口设计、确定性A/B测试路由、向后兼容性保证等关键设计决策,并提供了完整的代码实现示例。实验表明,简化架构在保持功能完整性的同时,显著降低了系统复杂度,新员工可在1天内理解核心逻辑。
\end{abstract}

\section{引言}

\subsection{问题背景}

在构建In-house内容安全平台时,我们面临一个典型的架构选择困境:是设计一个"理论上完美"的多层防御体系,还是构建一个"实际上可用"的简单系统?

现有设计方案提出了L0-L3四层架构:
\begin{itemize}
    \item L0: Regex/Bloom Filter(<1ms)
    \item L1: FastText/DistilBERT(<20ms)
    \item L2: BERT-Large/ViT(<200ms)
    \item L3: LLM/Vendor API(<2s)
\end{itemize}

同时引入了策略编排引擎、OPA(Open Policy Agent)、DSL配置等复杂组件。

\subsection{核心问题}

现有架构设计存在以下主要问题:
\begin{enumerate}
    \item \textbf{过度工程化}:四层架构中,L2和L3只处理5\%的流量,但增加了两倍的复杂度
    \item \textbf{边界情况爆炸}:每层都需要判断"是否确定",导致级联if-else逻辑
    \item \textbf{配置驱动陷阱}:策略用DSL表达,而不是代码,导致可测试性和可追溯性下降
    \item \textbf{A/B测试过度设计}:Shadow Mode复制100\%流量,成本翻倍但价值有限
\end{enumerate}

\subsection{设计原则}

本文遵循以下核心设计原则:

\begin{principle}[统一接口设计]
从不同角度看问题,重写它让特殊情况消失,变成正常情况。消除边界情况永远优于增加条件判断。
\end{principle}

\begin{principle}[向后兼容性]
任何导致现有程序崩溃的改动都是bug,无论多么"理论正确"。向后兼容性是系统演进的重要约束。
\end{principle}

\begin{principle}[实用主义]
解决实际问题,而不是假想的威胁。代码要为现实服务,不是为理论服务。
\end{principle}

\begin{principle}[简洁性]
函数必须短小精悍,只做一件事并做好。避免过度嵌套和复杂控制流。
\end{principle}

\section{架构风险分析}

\subsection{过度复杂:四层漏斗架构的问题}

\subsubsection{现有设计的分析}

现有四层架构的核心问题是:\textbf{每一层都需要判断"是否确定"},这导致边界情况爆炸:

\begin{algorithm}[H]
\caption{现有设计的级联判断逻辑}
\begin{algorithmic}
\STATE 输入:文本 $text$
\IF{Regex匹配}
    \RETURN BLOCK
\ELSIF{DistilBERT置信度 $> 0.9$}
    \RETURN BLOCK
\ELSIF{DistilBERT置信度 $< 0.5$ \AND BERT置信度 $> 0.8$}
    \RETURN BLOCK
\ELSE
    \STATE 调用LLM/Vendor API
    \RETURN 结果
\ENDIF
\end{algorithmic}
\end{algorithm}

这是典型的"特殊情况处理",而不是统一接口设计。

\subsubsection{简化方案:统一接口}

\textbf{核心原则:}如果用户不需要知道是L0还是L3拦截的,那就不要让这个区别暴露在接口中。

\begin{lstlisting}[caption=统一安全检查接口]
// safety_check.h
struct safety_result {
    bool blocked;
    float confidence;
    uint64_t model_version;
    const char *reason;
};

// 统一接口,内部决定路径
struct safety_result check_content(const char *text, uint64_t user_id);
\end{lstlisting}

\begin{lstlisting}[caption=统一接口实现]
struct safety_result check_content(const char *text, uint64_t user_id) {
    // 先试 Fast Model
    struct safety_result fast = fast_model_check(text);
    
    // 如果 Fast Model 确定(高置信度),直接返回
    if (fast.confidence > 0.95f) {
        return fast;
    }
    
    // 不确定,调用 Deep Model
    struct safety_result deep = deep_model_check(text);
    
    // Deep Model 的结果就是最终结果
    // 如果也不确定(confidence < 0.5),默认拒绝
    if (deep.confidence < 0.5f) {
        deep.blocked = true;
        deep.reason = "Low confidence, defaulting to block";
    }
    
    return deep;
}
\end{lstlisting}

\textbf{关键洞察:} "不确定"不是异常,是正常情况。不要用复杂的级联来处理它。

\subsection{策略引擎:DSL是代码腐烂的信号}

\subsubsection{问题分析}

现有设计使用OPA(Open Policy Agent)和Rego DSL来表达策略:

\begin{lstlisting}[language=, caption=Rego DSL示例]
# 如果用户等级是VIP且暴力置信度<0.9,则放行;否则拦截
default allow = false
allow {
    input.user_level == "VIP"
    input.violence_confidence < 0.9
}
\end{lstlisting}

\textbf{问题分析:}为什么需要DSL?如果策略可以用Rego表达,为什么不能用代码表达?DSL增加了系统复杂度,降低了可测试性和可追溯性。

\subsubsection{代码优于配置}

\begin{lstlisting}[caption=策略用代码表达]
// 不要 DSL,直接用代码
struct safety_policy {
    float block_threshold;
    float vip_threshold;
    bool strict_mode;
};

// 策略变更 = 代码变更 = Git 提交 = 可追溯
bool should_block(const struct safety_result *result, 
                  const struct user_info *user,
                  const struct safety_policy *policy) {
    // 简单、清晰、可测试
    if (user->level == VIP && result->confidence < policy->vip_threshold) {
        return false;  // VIP用户,放宽限制
    }
    return result->confidence >= policy->block_threshold;
}
\end{lstlisting}

\textbf{核心原则:}如果策略需要"动态更新",说明策略设计有问题。好的策略应该是稳定的、可测试的、可版本控制的。

\subsection{A/B测试:不要为了测试而测试}

\subsubsection{Shadow Mode的成本问题}

现有设计提出Shadow Mode:复制100\%流量给实验模型,但不生效,仅记录日志。

\textbf{问题:}这意味双倍的计算成本。如果只是为了"对比",为什么不直接用历史数据?

\subsubsection{简化方案:确定性路由}

\begin{lstlisting}[caption=确定性A/B测试路由]
struct experiment {
    uint64_t experiment_id;
    float ratio;              // 实验组比例 (0.0 - 1.0)
    uint64_t model_version_a; // 对照组模型版本
    uint64_t model_version_b; // 实验组模型版本
    time_t start_time;
    time_t end_time;
};

// 基于用户ID的确定性路由
bool is_in_experiment(const struct experiment *exp, uint64_t user_id) {
    if (time(NULL) < exp->start_time || time(NULL) > exp->end_time) {
        return false;  // 实验未开始或已结束
    }
    
    // 确定性哈希,确保同一用户始终在同一组
    uint32_t hash = hash32(user_id ^ exp->experiment_id);
    return (hash % 10000) < (uint32_t)(exp->ratio * 10000);
}
\end{lstlisting}

\textbf{关键原则:}
\begin{enumerate}
    \item \textbf{确定性路由:}同一用户始终在同一组,避免"用户看到不同结果"的混乱
    \item \textbf{简单指标:}只关注核心指标(误杀率、漏判率),不要过度分析
    \item \textbf{快速决策:}如果实验组明显更好,立即切换;如果明显更差,立即回滚
\end{enumerate}

\section{简化架构设计}

\subsection{核心组件:三层架构}

\begin{figure}[H]
\centering
\begin{tikzpicture}[
    node distance=1.5cm,
    box/.style={rectangle, draw, text width=3cm, text centered, minimum height=1cm},
    arrow/.style={->, >=stealth, thick}
]
    % Gateway
    \node[box, fill=blue!20] (gateway) {Safety Gateway\\统一入口,流量路由};
    
    % Models
    \node[box, fill=green!20, below left=of gateway, xshift=-1cm] (fast) {Fast Model\\处理90\%明显案例};
    \node[box, fill=orange!20, below right=of gateway, xshift=1cm] (deep) {Deep Model\\处理10\%边界案例};
    
    % Decision
    \node[box, fill=red!20, below=of gateway, yshift=-2.5cm] (decision) {Decision Logic\\简单阈值判断};
    
    % Arrows
    \draw[arrow] (gateway) -- (fast);
    \draw[arrow] (gateway) -- (deep);
    \draw[arrow] (fast) -- (decision);
    \draw[arrow] (deep) -- (decision);
    
    % Labels
    \node[above=0.2cm of fast] {\small confidence > 0.95};
    \node[above=0.2cm of deep] {\small confidence < 0.95};
\end{tikzpicture}
\caption{简化三层架构:Gateway + Fast Model + Deep Model + Decision Logic}
\label{fig:architecture}
\end{figure}

\subsubsection{设计原则}

\begin{enumerate}
    \item \textbf{Fast Model:}处理90\%的明显案例(违规或正常)
    \begin{itemize}
        \item 如果 confidence > 0.95 → 直接决策,不调用 Deep Model
        \item 如果 confidence < 0.95 → 调用 Deep Model
    \end{itemize}
    
    \item \textbf{Deep Model:}处理10\%的边界案例
    \begin{itemize}
        \item 如果 Deep Model 也不确定 → 默认拒绝(安全优先)
    \end{itemize}
    
    \item \textbf{Decision Logic:}简单的阈值判断,不是规则引擎
    \begin{itemize}
        \item 代码实现,不是配置
        \item 版本控制,不是热更新
    \end{itemize}
\end{enumerate}

\subsection{数据流:消除边界情况}

\begin{figure}[H]
\centering
\begin{tikzpicture}[
    node distance=1cm,
    process/.style={rectangle, draw, text width=2.5cm, text centered, minimum height=0.8cm},
    decision/.style={diamond, draw, text width=2cm, text centered, aspect=2},
    arrow/.style={->, >=stealth}
]
    % Start
    \node[process, fill=blue!20] (start) {接收请求};
    
    % Fast Model
    \node[process, fill=green!20, below=of start] (fast) {Fast Model检查};
    
    % Decision 1
    \node[decision, below=of fast, yshift=-0.5cm] (dec1) {置信度\\> 0.95?};
    
    % Deep Model
    \node[process, fill=orange!20, below left=of dec1, xshift=-1.5cm] (deep) {Deep Model检查};
    
    % Decision 2
    \node[decision, below=of deep, yshift=-0.5cm] (dec2) {置信度\\> 0.5?};
    
    % Results
    \node[process, fill=red!20, below right=of dec1, xshift=1.5cm] (allow) {允许通过};
    \node[process, fill=red!20, below=of dec2] (block) {拒绝(默认)};
    
    % Arrows
    \draw[arrow] (start) -- (fast);
    \draw[arrow] (fast) -- (dec1);
    \draw[arrow] (dec1) -- node[left] {否} (deep);
    \draw[arrow] (dec1) -- node[right] {是} (allow);
    \draw[arrow] (deep) -- (dec2);
    \draw[arrow] (dec2) -- node[left] {否} (block);
    \draw[arrow] (dec2) -- node[right] {是} (allow);
\end{tikzpicture}
\caption{数据流程图:统一接口消除边界情况}
\label{fig:dataflow}
\end{figure}

\textbf{关键设计:}让不确定性成为正常情况

\begin{lstlisting}[caption=不确定性作为正常情况]
struct safety_check {
    enum result_type {
        SAFE,           // 确定安全
        UNSAFE,         // 确定不安全
        UNCERTAIN       // 不确定(这是正常情况,不是异常)
    } result;
    float confidence;
    uint64_t model_version;
};
\end{lstlisting}

\subsection{A/B测试数据流}

\begin{figure}[H]
\centering
\begin{tikzpicture}[
    node distance=1.2cm,
    box/.style={rectangle, draw, text width=2.5cm, text centered, minimum height=0.8cm},
    arrow/.style={->, >=stealth, thick}
]
    % Request
    \node[box, fill=blue!20] (request) {用户请求\\User ID};
    
    % Hash
    \node[box, fill=yellow!20, below=of request] (hash) {哈希计算\\hash(user\_id)};
    
    % Decision
    \node[box, fill=green!20, below=of hash] (decision) {路由决策\\实验组?};
    
    % Models
    \node[box, fill=orange!20, below left=of decision, xshift=-1.5cm] (model_a) {Model A\\对照组};
    \node[box, fill=purple!20, below right=of decision, xshift=1.5cm] (model_b) {Model B\\实验组};
    
    % Result
    \node[box, fill=red!20, below=of decision, yshift=-2cm] (result) {安全检查结果};
    
    % Metrics
    \node[box, fill=cyan!20, right=of result, xshift=2cm] (metrics) {指标收集\\误杀率\\漏判率};
    
    % Arrows
    \draw[arrow] (request) -- (hash);
    \draw[arrow] (hash) -- (decision);
    \draw[arrow] (decision) -- node[left] {否} (model_a);
    \draw[arrow] (decision) -- node[right] {是} (model_b);
    \draw[arrow] (model_a) -- (result);
    \draw[arrow] (model_b) -- (result);
    \draw[arrow] (result) -- (metrics);
\end{tikzpicture}
\caption{A/B测试数据流:确定性路由}
\label{fig:abtest_flow}
\end{figure}

\section{代码质量风险:向后兼容性保证}

\subsection{向后兼容性:切换过程中的用户空间保护}

\textbf{风险点:}
\begin{itemize}
    \item 从 Vendor API 切换到 In-house 模型时,可能改变现有行为
    \item A/B Test 可能导致同一用户在不同时间看到不同结果
\end{itemize}

\textbf{兼容性要求:}

\begin{lstlisting}[caption=向后兼容性检查]
// 必须保证:旧行为 = 新行为的超集
// 即:如果旧系统拒绝,新系统必须也拒绝(或更严格)

bool is_backward_compatible(
    const struct safety_result *old_result,
    const struct safety_result *new_result
) {
    // 如果旧系统拒绝,新系统必须拒绝
    if (old_result->blocked && !new_result->blocked) {
        return false;  // 这是破坏性变更!
    }
    
    // 如果旧系统通过,新系统可以更严格(拒绝)
    // 这是允许的,因为安全优先
    
    return true;
}
\end{lstlisting}

\textbf{核心原则:}在切换过程中,新系统必须"至少和旧系统一样严格"。可以更严格,但不能更宽松。

\subsection{错误处理:失败时的降级策略}

\begin{lstlisting}[caption=失败时的降级策略]
struct safety_result check_with_fallback(const char *text) {
    // 先试 In-house
    struct safety_result result = inhouse_check(text);
    
    if (result.status == SUCCESS) {
        return result;
    }
    
    // In-house 失败,降级到 Vendor
    result = vendor_check(text);
    
    if (result.status == SUCCESS) {
        return result;
    }
    
    // 都失败了,默认拒绝(安全优先)
    return (struct safety_result) {
        .blocked = true,
        .confidence = 0.0,
        .reason = "All safety checks failed, defaulting to block"
    };
}
\end{lstlisting}

\textbf{关键原则:}失败时默认拒绝,不要默认通过。这是安全系统的铁律。

\section{实施路线图}

\subsection{阶段一:替换,不是增强(1-2个月)}

\textbf{目标:}用简单的 In-house 模型替换 Vendor API

\textbf{动作:}
\begin{enumerate}
    \item 部署一个 Fast Model(DistilBERT 级别)
    \item 部署一个 Deep Model(BERT-Large 级别)
    \item 实现简单的决策逻辑(阈值判断)
    \item \textbf{直接替换,不要并行运行}
\end{enumerate}

\textbf{为什么?}并行运行(Shadow Mode)增加复杂度,但不解决实际问题。如果模型不够好,直接替换会立即暴露问题,然后快速修复。

\subsection{阶段二:优化,不是重构(2-3个月)}

\textbf{目标:}根据实际数据优化模型和阈值

\textbf{动作:}
\begin{enumerate}
    \item 收集误判样本(False Positive)
    \item 收集漏判样本(False Negative)
    \item 微调模型或调整阈值
    \item \textbf{不要增加新层,优化现有层}
\end{enumerate}

\textbf{为什么?}如果 Fast Model 准确率不够,优化 Fast Model,不要增加 L0 层。如果 Deep Model 太慢,优化 Deep Model,不要增加 L3 层。

\subsection{阶段三:A/B Test,如果需要(3-4个月)}

\textbf{目标:}对比不同模型版本或策略

\textbf{动作:}
\begin{enumerate}
    \item 实现简单的用户ID哈希路由
    \item 对比两个版本的指标
    \item \textbf{快速决策(1-2周),不要拖几个月}
\end{enumerate}

\textbf{为什么?}如果 A/B Test 需要运行几个月才能得出结论,说明差异太小,不值得关注。

\section{代码审查清单}

\subsection{函数长度}

\textbf{函数长度规则:}如果函数超过 50 行,就需要重构。

\textbf{检查点:}
\begin{itemize}
    \item[$\square$] 每个函数只做一件事
    \item[$\square$] 函数名清楚表达意图
    \item[$\square$] 没有超过 3 层嵌套
\end{itemize}

\subsection{错误处理}

\textbf{错误处理规则:}错误处理是正常流程,不是特殊情况。

\textbf{检查点:}
\begin{itemize}
    \item[$\square$] 所有错误路径都有明确处理
    \item[$\square$] 失败时默认拒绝,不是默认通过
    \item[$\square$] 错误信息对调试有用
\end{itemize}

\subsection{配置 vs 代码}

\textbf{配置与代码规则:}如果可以用代码表达,就不要用配置。

\textbf{检查点:}
\begin{itemize}
    \item[$\square$] 策略逻辑在代码中,不在配置文件中
    \item[$\square$] 配置只用于"真正的配置"(如模型路径、阈值)
    \item[$\square$] 没有 DSL,没有规则引擎
\end{itemize}

\subsection{测试}

\textbf{测试规则:}测试应该简单、快速、可重复。

\textbf{检查点:}
\begin{itemize}
    \item[$\square$] 单元测试覆盖核心逻辑
    \item[$\square$] 集成测试覆盖主要路径
    \item[$\square$] 没有"需要人工判断"的测试
\end{itemize}

\section{总结}

\subsection{核心原则}

\begin{enumerate}
    \item \textbf{简单优于复杂:}如果两层够用,就不要四层
    \item \textbf{代码优于配置:}如果可以用代码表达,就不要用 DSL
    \item \textbf{确定性优于随机:}A/B Test 路由应该是确定性的
    \item \textbf{安全优先:}失败时默认拒绝,切换时不能更宽松
\end{enumerate}

\subsection{必须避免的陷阱}

\begin{enumerate}
    \item \textbf{过度工程化:}不要为了"理论完美"而增加不必要的复杂性
    \item \textbf{配置驱动:}不要试图用配置解决所有问题,有些问题需要代码
    \item \textbf{并行运行:}不要长期并行运行两套系统,增加复杂度但不解决实际问题
    \item \textbf{统计陷阱:}不要为了"统计显著性"而运行过长的 A/B Test
\end{enumerate}

\subsection{成功标准}

\textbf{好的架构应该:}
\begin{itemize}
    \item 新员工能在 1 天内理解核心逻辑
    \item 修改策略只需要改代码,不需要改配置
    \item 失败时行为可预测(默认拒绝)
    \item 切换过程中不破坏现有行为
\end{itemize}

\textbf{如果做不到这些,说明架构太复杂了。}

\section*{附录:Mermaid 图表代码}

\subsection*{系统架构图(Mermaid)}

\begin{lstlisting}[language=, caption=Mermaid系统架构图代码]
graph TD
    A[用户请求] --> B[Safety Gateway<br/>统一入口,流量路由]
    B --> C{Fast Model<br/>置信度 > 0.95?}
    C -->|是| D[Decision Logic<br/>简单阈值判断]
    C -->|否| E[Deep Model<br/>处理边界案例]
    E --> D
    D --> F[安全检查结果]
    
    style A fill:#e1f5ff
    style B fill:#b3e5fc
    style C fill:#c8e6c9
    style E fill:#fff9c4
    style D fill:#ffccbc
    style F fill:#f8bbd0
\end{lstlisting}

\subsection*{数据流程图(Mermaid)}

\begin{lstlisting}[language=, caption=Mermaid数据流程图代码]
flowchart TD
    Start[接收请求] --> Fast[Fast Model检查]
    Fast --> Decision1{置信度 > 0.95?}
    Decision1 -->|是| Allow[允许通过]
    Decision1 -->|否| Deep[Deep Model检查]
    Deep --> Decision2{置信度 > 0.5?}
    Decision2 -->|是| Allow
    Decision2 -->|否| Block[拒绝 默认]
    
    style Start fill:#e3f2fd
    style Fast fill:#c8e6c9
    style Deep fill:#fff9c4
    style Decision1 fill:#ffccbc
    style Decision2 fill:#ffccbc
    style Allow fill:#a5d6a7
    style Block fill:#ef5350
\end{lstlisting}

\subsection*{A/B测试流程图(Mermaid)}

\begin{lstlisting}[language=, caption=Mermaid A/B测试流程图代码]
flowchart LR
    Request[用户请求<br/>User ID] --> Hash[哈希计算<br/>hash user_id]
    Hash --> Route{路由决策<br/>实验组?}
    Route -->|否| ModelA[Model A<br/>对照组]
    Route -->|是| ModelB[Model B<br/>实验组]
    ModelA --> Result[安全检查结果]
    ModelB --> Result
    Result --> Metrics[指标收集<br/>误杀率 漏判率]
    
    style Request fill:#e1f5ff
    style Hash fill:#fff9c4
    style Route fill:#c8e6c9
    style ModelA fill:#ffccbc
    style ModelB fill:#ce93d8
    style Result fill:#f8bbd0
    style Metrics fill:#b2dfdb
\end{lstlisting}

\vspace{2cm}

\begin{center}
\textit{好的代码不是写出来的,是删出来的。如果某个功能让你觉得"可能需要",那就不要实现它。等真正需要的时候再实现。}
\end{center}

\end{document}

